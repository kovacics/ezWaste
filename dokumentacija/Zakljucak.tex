\chapter{Zaključak i budući rad}
		
%		\textbf{\textit{dio 2. revizije}}\\
%		
%		 \textit{U ovom poglavlju potrebno je napisati osvrt na vrijeme izrade projektnog zadatka, koji su tehnički izazovi prepoznati, jesu li riješeni ili kako bi mogli biti riješeni, koja su znanja stečena pri izradi projekta, koja bi znanja bila posebno potrebna za brže i kvalitetnije ostvarenje projekta i koje bi bile perspektive za nastavak rada u projektnoj grupi.}
%		
%		 \textit{Potrebno je točno popisati funkcionalnosti koje nisu implementirane u ostvarenoj aplikaciji.}
	 
		 \section{Projektni zadatak}
		 
		 Naš projektni zadatak je bio napraviti web aplikaciju za upravljanje kućnim otpadom. Cijeli razvoj je bio podijeljen u dvije faze.
		 
		 \subsection{Prva faza}
		 
		 Prvi zadatak nam je svima bio oformiti tim od 7 članova. Nakon osnivanja tima, bilo je potrebno ravnomjerno podijeliti posao i da se uz to, svaki član tima, bavi zadatcima u kojima se najbolje snalazi i/ili koje bi želio (dodatno) naučiti.\\
		 
		 Prvi ciklus razvoja programske potpore i dokumentacije se sastojao od dva glavna dijela: 
		 \begin{itemize}
		 	\item Razvoj generičke funkcionalnosti, koja se sastojala od:
		 	\begin{itemize}
		 		\item Razvoja stranica koje služe za registraciju i prijavu te početnih stranica za svaku vrstu korisnika
		 		\item Autentikacije i autorizacije koje se događaju u pozadini
		 	\end{itemize}
	 	\item Pisanje dokumentacije koja se sastojala od:
	 	\begin{itemize}
	 		\item Opisa projektnog zadatka
	 		\item Specifikacije programske potpore
	 		\item Opisa arhitekture
	 	\end{itemize}
		 \end{itemize}
	 
	 	Tijekom prvog ciklusa izrade projekta, iskušali smo se, dakle, i u razvojnom, i u dokumentacijskom dijelu projekta. 
	 
	 \subsection{Druga faza}
	 
	 Druga faza nije bila neovisna o prvoj fazi. Sva dokumentacija koju smo napravili u prvoj fazi nam je koristila u drugoj, to su najviše bili model baze, dijagram razreda i popis funkcionalnih zahtjeva. Dobro obavljena prva faza projekta, olakšala nam je drugu fazu.\\
	 Većinu vremena u drugoj fazi članovi tima su potrošili upravo na implementaciju.\\
	 
	 Drugi ciklus razvoja programske potpore i dokumentacije se sastojao od tri glavna dijela: 
	 \begin{itemize}
	 	\item Implementacija svih potrebnih funkcionalnosti
	 	\begin{itemize}
	 		\item Izrada api-ja na backendu
	 		\item Izrada grafičkog sučelja na frontendu
	 	\end{itemize}
 		\item Testiranje rada web aplikacije
	 	\item Pisanje dokumentacije koja se sastojala od:
	 	\begin{itemize}
	 		\item Dijagrama stanja, aktivnosti, komponenti i razmještaja
	 		\item Popisa tehnologija i alata koje smo koristili
	 		\item Uputa za puštanje aplikacije u pogon
	 	\end{itemize}
	 \end{itemize}
	 
	 \section{Opći dojmovi}
	 Svi članovi tima su zadovoljni projektom i konačnim proizvodom. Uloženo je mnogo truda i rada, ali ako razmotrimo koliko smo naučili tijekom ovom projekta, te kako izgleda konačni proizvod, onda se svi slažemo da je bilo vrijedno.\\
	 
	 Radom na projektu naučili smo kako je raditi u timu na zajedničkom projektu, a većinu vremena komunicirati preko platformi kao što je Slack.\\
	 
	 Tijekom procesa razvoja, kako bi se ispoštovali krajnji rokovi i zahtjevi, bilo je potrebno redovito pratiti napredak podtimova i pojedinaca u razvojnom timu, te organizirano dijeliti posao po određenim tjednima. Stizanjem takvih rokova, naučili smo koliko je bitna dobra organizacija, i kako uspjeti napraviti dobru organizaciju.\\
	 
	 Naravno nije uvijek bilo sve sjajno tijekom projekta, i putem smo se sreli s mnogo problema, ali smo isto tako onda zajedno sudjelovali u traženju rješenja za svaki od tih problema. 
	 	
		
		\eject 